\documentclass[12pt]{article}

\usepackage{geometry}
\geometry{
a4paper,
total={170mm,257mm},
left=20mm,
top=10mm,
}

\usepackage[utf8]{inputenc}
\usepackage[russian]{babel}

\usepackage{amsmath, amssymb}

\newcommand{\vverh}{\vspace*{-0.3cm}}

\begin{document}

\subsection*{Замена переменных в одномерном интеграле}

Рассмотрим интеграл от функции одной переменной в пределах от $0$ до $+\infty$. Осуществим замену переменных, позволяющую интегрировать на интервале $[0, 1]$:
\vverh
\begin{gather}
		\int\limits_{0}^{\infty} f(x) d x = \left[ y = \frac{2}{\pi} \arctg x \right] = \frac{\pi}{2} \int\limits_{0}^{1} \left[ 1 + \tg^2 \left( \frac{\pi}{2} y \right) \right] f \left( \tg \left( \frac{\pi}{2} y \right) \right) dy \notag 
\end{gather}

Введем дополнительное обозначение, чтобы упростить полученное выражение:
\vverh
\begin{gather}
		x = \tg \left( \frac{\pi}{2} y \right) \notag \\
	I = \frac{\pi}{2} \int\limits_{0}^{1} \left[ 1 + x^2 \right] f (x) dy \notag
\end{gather}

\subsection*{Замена переменных в двумерном интеграле}

Рассмотрим двумерный интеграл, в котором интегрирование по одной переменной происходит вдоль луча $[0, \infty)$, а по второй -- по отрезку $\left[ 0, \pi \right]$ (именно такого типа интеграл фигурирует в выражении для ВВК для системы Ar-CO$_2$).
\vverh
\begin{gather}
		\int\limits_{0}^{\infty} dx \int\limits_{0}^{\pi} f(x, y) dy \notag
\end{gather}

Для того, чтобы область интегрирования свести к единичному квадрату, можно осуществить следующую замену переменных:
\vverh
\begin{gather}
	\begin{aligned}
			X &= \frac{2}{\pi} \arctg x \\
			Y &= \frac{1}{\pi} y
	\end{aligned}
	\quad \quad \quad 
	\begin{aligned}
			x &= \tg \left( \frac{\pi}{2} X \right) \\ 
			y &=\pi Y
	\end{aligned}
	\notag
\end{gather}

Учитывая производные, вылазящие при замене переменных, приходим к следующему выражению:
\vverh
\begin{gather}
		\int\limits_{0}^{\infty} dx \int\limits_{0}^{\pi} f(x, y) dy = \frac{\pi^2}{2} \int\limits_0^1 d X \int\limits_0^1 f \left( \tg \left( \frac{\pi}{2} X \right), \pi Y \right) \left( 1 + \tg^2 \left( \frac{\pi}{2} X \right) \right) d Y \notag
\end{gather}

С вычислительной точки зрения проще пересчитывать старые переменные $\left( x, y \right)$ через новые $\left( X, Y \right)$, чтобы рассчитать значение подынтегрального выражения. 
\vverh
\begin{gather}
		\frac{\pi^2}{2} \int\limits_0^1 d X \int\limits_0^1 f \left( \tg \left( \frac{\pi}{2} X \right), \pi Y \right) \left( 1 + \tg^2 \left( \frac{\pi}{2} X \right) \right) d Y = \frac{\pi^2}{2} \int\limits_0^1 d X \int\limits_0^1 f \left( x, y \right) (1 + x^2 ) d Y \notag
\end{gather}


\end{document}
